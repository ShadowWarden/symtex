% astr5400/hw2/hw2.tex
%
% Omkar H. Ramachandran
% omkar.ramachandran@colorado.edu
%
% LaTeX writeup of HW2
%

\documentclass[english]{article}
\usepackage[T1]{fontenc}
\usepackage[latin9]{inputenc}
\usepackage{geometry}
\geometry{verbose,tmargin=1.5in,bmargin=1.5in,lmargin=1.5in,rmargin=1.5in}
\usepackage{babel}
\newcommand{\GeV}{\,{\rm GeV}}
\usepackage{graphicx}
\graphicspath{{./plots/}}
\usepackage{hyperref}
\usepackage{listings}
\usepackage{color}
\usepackage{amsmath}
\lstdefinestyle{custompy}{
  belowcaptionskip=1\baselineskip,
  breaklines=true,
  frame=L,
  xleftmargin=\parindent,
  language=Python,
  showstringspaces=false,
  basicstyle=\footnotesize\ttfamily,
  keywordstyle=\bfseries\color{green},
  commentstyle=\itshape\color{red},
  identifierstyle=\color{black},
  stringstyle=\color{blue},
}

\newcommand{\eval}[1]{\immediate\write18{python3 symtex.py '#1' > intermediate.txt}\input{intermediate.txt}}


\lstset{escapechar=@,style=custompy}

\begin{document}

\title{SymTEX: HackCU Demo}

\author{Omkar H. Ramachandran, Aidan Bohenick, Anthony Tracy, Kellie Gadeken}
\maketitle

$$ \intop{\frac{1}{x} dx} = \eval{\intop{frac{1}{x} dx}} $$

$$ \intop{\frac{2}{\alpha} d\alpha} = \eval{\intop{frac{2}{alpha} dalpha}} $$

$$ \intop{\sin{\alpha_q} d\alpha_q} = \eval{\intop{sin{alpha_q} dalpha_q}} $$

$$ \intop{\frac{x_1}{5!} dx_2} = \eval{\intop{frac{x_1}{5!} dx_2}} $$

$$ \intop{\tanh(x) dx} = \eval{\intop{tanh(x) dx}} $$

$$ \intop{\exp{\frac{x^2}{2}} dx} = \eval{\intop{exp{frac{x^2}{2}} dx}} $$

$$ \intop{\frac{1}{\sqrt{1 - x^2}} dx} = \eval{\intop{frac{1}{sqrt{1 - x^2}} dx}} $$

$$ \intop{x^{-\frac{5}{2}} dx} = \eval{\intop{x^{-frac{5}{2}} dx}} $$

$$ \intop{\frac{1}{\sqrt{6}} \exp{(j k x)} dx} = \eval{\intop{frac{1}{sqrt{6}} exp{(j k x)} dx}}$$

$$ \intop{(a_1 x^3 + a_2 x^2 + a_3 x ) dx} = \eval{\intop{a_1 x^3 + a_2 x^2 + a_3 x dx}}$$

$$ \intop{(\log(\frac{1}{x}))^2 dx} = \eval{\intop{(log(frac{1}{x}))^2 dx}} $$

$$ \intop{\cos(a \alpha_1) \cosh(b \alpha_1) d\alpha_1} = \eval{\intop{cos(a alpha_1) cosh(b alpha_1) dalpha_1}} $$











\end{document}
